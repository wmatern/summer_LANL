\section{Introduction}
\label{intro}

The earthquake and resulting tsunami in Japan caused a release of radioactive material into the pacific ocean. This radioactive waste was carried by the Kuroshio current into the North Pacific current and is making its way to the US coastline. These currents can produce sharp flow discontinuities with the surrounding ocean water. The idea of tracking these flows and accurately resolving the  interfaces between the radioactive and non-radioactive fluids provides a motivation for resolving contact discontinuities.

Many classic techniques have been developed to analyze contact discontinuities. The most simple scheme is a first order upwind method. First order schemes can be proven to preserve their order across discontinuities (monotonicity). Although this method is monotone, a large amount of diffusion error is introduced at each time step. Therefore, higher order methods have been developed to solve problems without introducing such large errors. The difficulty with higher order methods is that, according to Godunov's Theorem, they cannot be monotone across discontinuities. Therefore, limiters have been introduced and will be discussed in Section \ref{Lax}.

Several higher order finite difference methods (the Lax-Wendroff method with limiters, a Multidimensional Positive Definite Advection Transport Algorithm (MPDATA) method and a Monotone Upstream Scheme for Conservation Laws (MUSCL) with an Artificial Compression Method (ACM)) as well as the finite element Streamlined Upwind Petrov-Galerkin (SUPG) method are considered and compared for their effectiveness. These methods were applied to solve the 1D advection equation with three types of waves: a square wave, a triangular wave and an exponential smooth wave. Based on these results, we were able to conclude that the MUSCL scheme with artificial compression produces the sharpest interfaces but can inappropriately steepen the solution. The SUPG method seems to represent a good balance between stability and interface sharpness without any inappropriate steepening but is a non-conservative method. However, for solutions with discontinuities, the MUSCL scheme is still superior. Each of these methods is detailed in Section \ref{Methods} followed by a brief description of the results from the different methods in Section \ref{results}. We then discuss the conclusions and future directions we hope to take with the research in Section \ref{conclusions}.

\subsection{Introduction to GPUs}
\label{GPU}
Graphics Processing Units (GPU) are typically used for accelerating graphics intensive applications. While these devices have been optimized for such applications, their ability to process significant data is parallel can also be harnessed for more general use. Because of recent advances in software, general purpose computing on graphics processing units (GPGPU) is an alternative to using additional CPUs for executing a program. GPGPU has the additional economic advantage of providing a greater number of floating point operations per second (FLOPS) per dollar. 

