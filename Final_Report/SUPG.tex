\subsection{SUPG}
\label{SUPG}
In the finite element method a set of trial functions are assumed to represent the solution of a differential equation over a region. These trial functions have certain degrees of freedom that must be solved for. The differential equation is multiplied by a test function and the new equation is then integrated over the region that the original differential equation is to be solved on. This generally yields one unique algebraic equations for each test function used. The resulting system of algebraic equations are then solved for the degrees of freedom. 
In Galerkin approaches, the test functions are assumed to be the same as the trial functions. However, when sharp gradients are present Galerkin methods produce highly oscillatory solutions. This can be helped in the case where the differential equation is an advection/diffusion equation by introducing the more general Petrov-Galerkin (PG) approach in which the test functions can be different from the trial functions. The streamline upwind Petrov-Galerkin (SUPG) approach chooses a specific set of test functions which can be observed to significantly reduce oscillations in the solution. For the 2D advection equation these test functions are the same as the trial functions with two additional terms:
\begin{equation}
W = N + \frac{\alpha h}{2}\left(\frac{u}{\sqrt{u^2+v^2}}\frac{\partial N}{\partial x} + \frac{v}{\sqrt{u^2+v^2}}\frac{\partial N}{\partial y}\right)
\end{equation}
Where $W$ is a test function, $N$ is one of the trial functions, $\alpha$ is an arbitrary parameter usually taken to be 1 for pure advection, $h$ is a 1D length of an element, and $u$-$v$ are the velocity in the $x$-$y$ direction, respectively.
