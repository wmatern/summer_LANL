\subsection{Lax-Wendroff}
\label{Lax}

The Lax-Wendroff method is a finite difference method designed for hyperbolic PDEs that is second order accurate in both space and time. This scheme can be understood as a two-step method. On the first step the fluxes at a half step in both space and time are calculated. On the second step of the method the half-step fluxes are used to calculate the new value of the conserved variable at each point in the domain. This scheme works well for smooth data but can run into trouble when the solution develops a discontinuity. When this occurs, the solution becomes highly oscillatory around the discontinuity.

It has been shown by numerous authors \cite{Rider,Leveque} that there is a simple fix for oscillations around discontinuities. First order schemes for hyperbolic PDEs (such as upwind finite differencing) do not lead to oscillations like higher order schemes (Lax-Wendroff). Flux limiters work by interpolating between the flux calculated by a first order scheme and the flux calculated by a higher order scheme. Oscillations near a discontinuity can be eliminated by making the approximation mimic a first order scheme near a steep gradient using a flux limiter and while higher order accuracy is achieved in smoother regions. 

It has been mathematically shown that certain formulas for flux limiters in linear PDEs lead to schemes which are monotonicity preserving. This means that if the solution is monotonic at a certain time step it will remain monotonic at the next time step. This property is equivalent to saying that the scheme will not introduce new extrema and that the scheme is Total Variation Diminishing (TVD). We explored several limiters to preserve the monotonicity of our solutions and settled on the Superbee method, which produces the sharpest possible gradients while still being TVD.