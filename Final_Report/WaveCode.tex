\section{Wave Code}
\label{WaveCode}
\subsection{GPU Architecture}
Don't forget to mention that our code uses OpenCL but CUDA is another language that only works on NVIDIA.
\subsection{Model Equations and Method}
The GPU enabled code we have been working with models the shallow water equations using a Lax-Wendroff method with a Minmod TVD Limiter, as described in Section \ref{Lax}. This simplification of the Euler Equations assumes that the length of waves is far greater than the depth of the fluid. This allows for an incompressible model:
\[
\frac{\partial h}{\partial t} + \frac{\partial(uh)}{\partial x} + \frac{\partial(vh)}{\partial y} = 0
\] \[
\frac{\partial (uh)}{\partial t} + \frac{\partial(u^2h+\nicefrac{1}{2} g h^2)}{\partial x} + \frac{\partial(uvh)}{\partial y} = 0
\] \[
\frac{\partial (vh)}{\partial t} + \frac{\partial(uvh)}{\partial x}+ \frac{\partial(v^2h+\nicefrac{1}{2} g h^2)}{\partial y}  = 0
\]
where $h$ is the height, $u$ is the velocity in the $x$-direction, $v$ is the velocity in the $y$-direction and $g$ is the gravitational acceleration. With the introduction of an advection equation for a passive tracer concentration, $\psi$:
\[
\frac{\partial \psi}{\partial t} + u\frac{\partial \psi}{\partial x} + v\frac{\partial \psi}{\partial y} = 0
\]
we can look at our motivational problem, namely the advection of radioactive water across the Pacific Ocean more closely in the future.