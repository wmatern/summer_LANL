\subsection{Introduction to GPUs}
\label{GPU}
Graphics Processing Units (GPU) are typically used for accelerating graphics intensive applications. While these devices have been optimized for such applications, their ability to process significant data in parallel can also be harnessed for more general use. Because of recent advances in software, general purpose computing on graphics processing units (GPGPU) is an alternative to using additional CPUs for executing a program. The two main languages used to program on GPUs are CUDA and OpenCL. While CUDA has more libraries available, it has the disadvantage of only being able to be used on NVIDIA GPU cards while OpenCL can be used with any programmable GPU. GPGPU has the additional economic advantage of providing a greater number of floating point operations per second (FLOPS) per dollar. A GPU enabled code was developed to solve the shallow water equations and is described in Section \ref{WaveCode}.
